% !TeX spellcheck = pl_PL
%
%%%%%%%%%%%%%%%%%%%%%%%%%%%%%%%%%%%%%%%%%%
%                                        %
% Szablon pracy dyplomowej magisterskiej % 
% zgodny  z aktualnymi  przepisami  SZJK %
%                                        %
%%%%%%%%%%%%%%%%%%%%%%%%%%%%%%%%%%%%%%%%%%
%                                        %
%  (c) Krzysztof Simiński, 2018-2022     %
%                                        %
%%%%%%%%%%%%%%%%%%%%%%%%%%%%%%%%%%%%%%%%%%
%                                        %
% Najnowsza wersja szablonów jest        %
% podstępna pod adresem                  %
% github.com/ksiminski/polsl-aei-theses  %
%                                        %
%%%%%%%%%%%%%%%%%%%%%%%%%%%%%%%%%%%%%%%%%%
%
%
% Projekt LaTeXowy zapewnia odpowiednie formatowanie pracy,
% zgodnie z wymaganiami Systemu zapewniania jakości kształcenia.
% Proszę nie zmieniać ustawień formatowania (np. fontu,
% marginesów, wytłuszczeń, kursywy itd. ).
%
% Projekt można kompilować na kilka sposobów.
%
% 1. kompilacja pdfLaTeX
%
% Jeżeli w pracy używany jest pakiet minted do formatowania
% kodów źródłowych należy kompilować w następujący sposób:
%
% pdflatex -shell-escape praca
% biber                  praca
% pdflatex -shell-escape praca
% pdflatex -shell-escape praca 
%
% Jeżeli pakiet minted nie jest wykorzystywany, można zakomentować
% import tego pakietu \usepackage{minted} i użycie w kodzie pracy.
% Wtedy wystarczy prosta kompilacja 
%
% pdflatex praca
% biber    praca
% pdflatex praca
% pdflatex praca 
%
%
% 2. kompilacja XeLaTeX
%
% Kompilatacja przy użyciu XeLaTeXa różni się tym, że na stronie
% tytułowej używany jest font Calibri. Wymaga to jego uprzedniego
% zainstalowania. Podobnie jak w przypadku pdfLaTeXa użycie
% pakietu minted wymaga kompilacji jak poniżej:
%
% xelatex -shell-escape praca
% biber                 praca
% xelatex -shell-escape praca
% xelatex -shell-escape praca 
%
% Bez pakietu minted kompilacja jest trochę prostsza:
%
% xelatex praca
% biber   praca
% xelatex praca
% xelatex praca 
%
%
% dokumentacja pakietów do kodów źródłowych:
% https://ctan.org/pkg/minted
% https://ctan.org/pkg/listings
%
%%%%%%%%%%%%%%%%%%%%%%%%%%%%%%%%%%%%%%%%%%%%%%%%%%%%%
% W przypadku pytań, uwag, proszę pisać na adres:   %
%      krzysztof.siminski(małpa)polsl.pl            %
%%%%%%%%%%%%%%%%%%%%%%%%%%%%%%%%%%%%%%%%%%%%%%%%%%%%%

%%%%%%%%%%%%%%%%%%%%%%%%%%%%%%%%%%%%%%%%%%%%%%%%%%%%%%%%%%%%%%%%%%%%%%%%%
% Proszę wpisać swoje dane w poniższych definicjach.

% TODO
\newcommand{\FirstName}{Imię}
\newcommand{\Surname}{Nazwisko}
\newcommand{\Supervisor}{$\langle$tytuł lub stopień naukowy oraz imię i nazwisko$\rangle$}
\newcommand{\Title}{Tytuł pracy dyplomowej magisterskiej}
\newcommand{\TitleAlt}{Thesis title in English}
\newcommand{\Program}{$\langle$wpisać właściwy$\rangle$}
\newcommand{\Specialisation}{$\langle$wpisać właściwą$\rangle$}
\newcommand{\Id}{$\langle$wpisać właściwy$\rangle$}
\newcommand{\Departament}{$\langle$wpisać właściwą$\rangle$}

% Jeżeli został wyznaczony promotor pomocniczy lub opieku proszę go/ją wpisać ...
\newcommand{\opiekun}{$\langle$stopień naukowy imię i nazwisko$\rangle$}
% ... w przeciwnym razie proszę zostawić puste miejsce jak poniżej:
%\newcommand{\opiekun}{} % brak promotowa pomocniczego / opiekuna

% koniec fragmentu do modyfikacji
%%%%%%%%%%%%%%%%%%%%%%%%%%%%%%%%%%%%%%%%%%


%%%%%%%%%%%%%%%%%%%%%%%%%%%%%%%%%%%%%%%%%%%%%%%
%                                             %
% PROSZĘ NIE MODYFIKOWAĆ PONIŻSZYCH USTAWIEŃ! %
%                                             %
%%%%%%%%%%%%%%%%%%%%%%%%%%%%%%%%%%%%%%%%%%%%%%%



\documentclass[a4paper,twoside,12pt]{book}
\usepackage[utf8]{inputenc}                                      
\usepackage[T1]{fontenc}  
\usepackage{amsmath,amsfonts,amssymb,amsthm}
\usepackage[british,polish]{babel} 
\usepackage{indentfirst}



\usepackage{ifxetex}

\ifxetex
	\usepackage{fontspec}
	\defaultfontfeatures{Mapping=tex—text} % to support TeX conventions like ``——-''
	\usepackage{xunicode} % Unicode support for LaTeX character names (accents, European chars, etc)
	\usepackage{xltxtra} % Extra customizations for XeLaTeX
\else
	\usepackage{lmodern}
\fi



\usepackage[margin=2.5cm]{geometry}
\usepackage{graphicx} 
\usepackage{hyperref}
\usepackage{booktabs}
\usepackage{tikz}
\usepackage{pgfplots}
\usepackage{mathtools}
\usepackage{geometry}
\usepackage[page]{appendix} % toc,
\renewcommand{\appendixtocname}{Dodatki}
\renewcommand{\appendixpagename}{Dodatki}
\renewcommand{\appendixname}{Dodatek}

\usepackage{csquotes}
\usepackage[natbib=true]{biblatex}
\bibliography{biblio/bibliografia}

\usepackage{ifmtarg}   % empty commands  

\usepackage{setspace}
\onehalfspacing


\frenchspacing

%%%%%%%%%%%%%%%%%%%%%%%%%%%%%%%%%%%%%%%%%%%%%%%%%%%%%%%%%%%%%%%%%%%%%
% listingi i fragmentu kodu źródłowego 
% pakiet: listings lub minted
% % % % % % % % % % % % % % % % % % % % % % % % % % % % % % % % % % % 

% biblioteka listings
\usepackage{listings}
\lstset{%
language=C++,%
commentstyle=\textit,%
identifierstyle=\textsf,%
keywordstyle=\sffamily\bfseries, %\texttt, %
%captionpos=b,%
tabsize=3,%
frame=lines,%
numbers=left,%
numberstyle=\tiny,%
numbersep=5pt,%
breaklines=true,%
%morekeywords={descriptor_gaussian,descriptor,partition,fcm_possibilistic,dataset,my_exception,exception,std,vector},%
escapeinside={@*}{*@},%
}

% % % % % % % % % % % % % % % % % % % % % % % % % % % % % % % % % % % 
% pakiet minted
\usepackage{minted}

% pakiet wymaga specjalnego kompilowania:
% pdflatex -shell-escape praca

%%%%%%%%%%%%%%%%%%%%%%%%%%%%%%%%%%%%%%%%%%%%%%%%%%%%%%%%%%%%%%%%%%%%%


%%%% TODO LIST GENERATOR %%%%%%%%%

\usepackage{color}
\definecolor{brickred}      {cmyk}{0   , 0.89, 0.94, 0.28}

\makeatletter \newcommand \kslistofremarks{\section*{Uwagi} \@starttoc{rks}}
  \newcommand\l@uwagas[2]
    {\par\noindent \textbf{#2:} %\parbox{10cm}
{#1}\par} \makeatother


\newcommand{\ksremark}[1]{%
{%\marginpar{\textdbend}
{\color{brickred}{[#1]}}}%
\addcontentsline{rks}{uwagas}{\protect{#1}}%
}

\newcommand{\comma}{\ksremark{przecinek}}
\newcommand{\nocomma}{\ksremark{bez przecinka}}
\newcommand{\styl}{\ksremark{styl}}
\newcommand{\ortografia}{\ksremark{ortografia}}
\newcommand{\fleksja}{\ksremark{fleksja}}
\newcommand{\pauza}{\ksremark{pauza `--', nie dywiz `-'}}
\newcommand{\kolokwializm}{\ksremark{kolokwializm}}
\newcommand{\cudzyslowy}{\ksremark{,,polskie cudzysłowy''}}

%%%%%%%%%%%%%% END OF TODO LIST GENERATOR %%%%%%%%%%%

%%%%%%%%%%%% ZYWA PAGINA %%%%%%%%%%%%%%%
% brak kapitalizacji zywej paginy
\usepackage{fancyhdr}
\pagestyle{fancy}
\fancyhf{}
\fancyhead[LO]{\nouppercase{\it\rightmark}}
\fancyhead[RE]{\nouppercase{\it\leftmark}}
\fancyhead[LE,RO]{\it\thepage}


\fancypagestyle{tylkoNumeryStron}{%
   \fancyhf{} 
   \fancyhead[LE,RO]{\it\thepage}
}

\fancypagestyle{bezNumeracji}{%
   \fancyhf{} 
   \fancyhead[LE,RO]{}
}

\fancypagestyle{NumeryStronNazwyRozdzialow}{%
   \fancyhf{} 
   \fancyhead[LE]{\nouppercase{\FirstName\ \Surname}}
   \fancyhead[RO]{\nouppercase{\leftmark}} 
   \fancyfoot[CE, CO]{\thepage}
}


%%%%%%%%%%%%% OBCE WTRETY  
\newcommand{\obcy}[1]{\emph{#1}}
\newcommand{\ang}[1]{{\selectlanguage{british}\obcy{#1}}}
%%%%%%%%%%%%%%%%%%%%%%%%%%%%%

% polskie oznaczenia funkcji matematycznych
\renewcommand{\tan}{\operatorname {tg}}
\renewcommand{\log}{\operatorname {lg}}

% jeszcze jakies drobiazgi

\newcounter{stronyPozaNumeracja}

%%%%%%%%%%%%%%%%%%%%%%%%%%% 
\usepackage{xstring}
\newcommand{\printOpiekun}[1]{%		

    \StrLen{\opiekun}[\mystringlen]
    \ifthenelse{\mystringlen > 0}%
    {%
       {\large{\bfseries OPIEKUN, PROMOTOR POMOCNICZY}\par}
       
       {\large{\bfseries \opiekun}\par}
    }%
    {}
} 
%
%%%%%%%%%%%%%%%%%%%%%%%%%%%%%%%%%%%%%%%%%%%%%%
 
% Proszę nie modyfikować poniższych definicji!
\author{\FirstName\ \Surname}
\newcommand{\Author}{\FirstName\ \MakeUppercase{\Surname}}
\newcommand{\Type}{PRACA MAGISTERSKA}
\newcommand{\Faculty}{Wydział Automatyki, Elektroniki i Informatyki} 
\newcommand{\Polsl}{Politechnika Śląska}
\newcommand{\Logo}{graf/politechnika_sl_logo_bw_pion_pl.pdf}
\newcommand{\LeftId}{Nr albumu}
\newcommand{\LeftProgram}{Kierunek}
\newcommand{\LeftSpecialisation}{Specjalność}
\newcommand{\LeftSUPERVISOR}{PROWADZĄCY PRACĘ}
\newcommand{\LeftDEPARTMENT}{KATEDRA}
%%%%%%%%%%%%%%%%%%%%%%%%%%%%%%%%%%%%%%%%%%%%%%

 % Proszę nie modyfikować pliku settings.tex

% Poniżej można dodać słowa kluczowe dla
\lstset{%
morekeywords={string,exception,std,vector},% słowa kluczowe rozpoznawane przez pakiet listings
}


%%%%%%%%%%%%%%%%%%%%%%%%%%%%%%%%%%%%%%%%   


\begin{document}
%\kslistofremarks 

%%%%%%%%%%%%%%%%%%%%%%%%%%%%%%%%%%%%%%%%%%%%%%%
%                                             %
% PROSZĘ NIE MODYFIKOWAĆ PONIŻSZYCH USTAWIEŃ! %
%                   main                          %
%%%%%%%%%%%%%%%%%%%%%%%%%%%%%%%%%%%%%%%%%%%%%%%


%%%%%%%%%%%%%%%%%%  TITLE PAGE %%%%%%%%%%%%%%%%%%%
\pagestyle{empty}
{
	\newgeometry{top=1.5cm,%
	             bottom=2.5cm,%
	             left=3cm,
	             right=2.5cm}
 
	\ifxetex 
	  \begingroup
	  \setsansfont{Calibri}
	   
	\fi 
	 \sffamily
	\begin{center}
	\includegraphics[width=50mm]{\Logo}
	 
	
	{\Large\bfseries\Type\par}
	
	\vfill  \vfill  
			 
	{\large\Title\par}
	
	\vfill  
		
	{\large\bfseries\Author\par}
	
	{\normalsize\bfseries \LeftId: \Id}
	
	\vfill  		
 
	{\large{\bfseries \LeftProgram:} \Program\par} 
	
	{\large{\bfseries \LeftSpecialisation:} \Specialisation\par} 
	 		
	\vfill  \vfill 	\vfill 	\vfill 	\vfill 	\vfill 	\vfill  
	 
	{\large{\bfseries \LeftSUPERVISOR}\par}
	
	{\large{\bfseries \Supervisor}\par}
				
	{\large{\bfseries \LeftDEPARTMENT\ \Departament} \par}
		
	{\large{\bfseries \Faculty}\par}
		
	\vfill  \vfill  

    	
    \printOpiekun{\Consultant}
    
	\vfill  \vfill  
		
    {\large\bfseries  Gliwice \the\year}

   \end{center}	
       \ifxetex 
       	  \endgroup
       \fi
	\restoregeometry
}
  
  % Proszę nie modyfikować pliku titlepage.tex

\cleardoublepage
 
\rmfamily\normalfont
\pagestyle{empty}

  
%%% No to zaczynamy pisać pracę :-) %%%%

% TODO
\subsubsection*{Tytuł pracy} 
\Title

\subsubsection*{Streszczenie}  
(Streszczenie pracy – odpowiednie pole w systemie APD powinno zawierać kopię tego streszczenia.)

\subsubsection*{Słowa kluczowe} 
(2-5 slow (fraz) kluczowych, oddzielonych przecinkami)

\subsubsection*{Thesis title} 
\begin{otherlanguage}{british}
\TitleAlt
\end{otherlanguage}

\subsubsection*{Abstract} 
\begin{otherlanguage}{british}
(Thesis abstract – to be copied into an appropriate field during an electronic submission – in English.)
\end{otherlanguage}
\subsubsection*{Key words}  
\begin{otherlanguage}{british}
(2-5 keywords, separated by commas)
\end{otherlanguage}

 % informacje redakcyjne


%%%%%%%%%%%%%%%%%% SPIS TRESCI %%%%%%%%%%%%%%%%%%%%%%
%\pagenumbering{Roman}
\thispagestyle{empty}
\tableofcontents
\thispagestyle{empty}

%%%%%%%%%%%%%%%%%%%%%%%%%%%%%%%%%%%%%%%%%%%%%%%%%%%%%
\setcounter{stronyPozaNumeracja}{\value{page}}
\mainmatter
\pagestyle{empty}
 
\cleardoublepage

\pagestyle{NumeryStronNazwyRozdzialow}

%%%%%%%%%%%%%% wlasciwa tresc pracy %%%%%%%%%%%%%%%%%

% TODO
\chapter{Wstęp}

%\begin{itemize}
%\item wprowadzenie w problem/zagadnienie 
%\item osadzenie problemu w dziedzinie 
%\item cel pracy 
%\item zakres pracy 
%\item zwięzła charakterystyka rozdziałów 
%\end{itemize}

  % wstęp

% TODO
\chapter{[Tytuł rozdziału]}

Odwołania do literatury: 
książek \cite{bib:ksiazka},
artykułów w czasopismach \cite{bib:artykul}, 
materiałów konferencyjnych \cite{bib:konferencja}
i stron www \cite{bib:internet}.

Równania powinny być numerowane
\begin{align}
y = \frac{\partial x}{\partial t}
\end{align}

%\chapter{[Analiza tematu]}
%
%
%\begin{itemize}
%\item analiza tematu
%\item wprowadzenie do dziedziny (\ang{state of the art}) – sformułowanie problemu, 
%\item poszerzone studia literaturowe, przegląd literatury tematu (należy wskazać źródła wszystkich informacji zawartych w pracy)
%\item opis znanych rozwiązań, algorytmów, osadzenie pracy w kontekście
%\item Tytuł rozdziału jest często zbliżony do tematu pracy. 
%\item Rozdział jest wysycony cytowaniami do literatury \cite{bib:artykul,bib:ksiazka,bib:konferencja}. 
%Cytowanie książki \cite{bib:ksiazka}, artykułu w czasopiśmie \cite{bib:artykul}, artykułu konferencyjnego \cite{bib:konferencja} lub strony internetowej \cite{bib:internet}.
%\end{itemize}
 % [Analiza tematu] 

% TODO

\chapter{[Tytuł rozdziału]}

tekst

\section{[Tytuł podrozdziału]}

\section{[Tytuł podrozdziału]}

W całym dokumencie powinny znajdować się odniesienia do zawartych w nim ilustracji (rys. \ref{fig:2}).

\begin{figure}
\centering
\begin{tikzpicture}
\begin{axis}[
    y tick label style={
        /pgf/number format/.cd,
            fixed,   % po zakomentowaniu os rzednych jest indeksowana wykladniczo
            fixed zerofill, % 1.0 zamiast 1
            precision=1,
        /tikz/.cd
    },
    x tick label style={
        /pgf/number format/.cd,
            fixed,
            fixed zerofill,
            precision=2,
        /tikz/.cd
    }
]
\addplot [domain=0.0:0.1] {rnd};
\end{axis} 
\end{tikzpicture}
\caption{Wykres przebiegu funkcji.} % Podpis jest zawsze POD rysunkiem.
\label{fig:2}
\end{figure}

Tekst dokumentu powinien również zawierać odniesienia do tabel (tab. \ref{id:tab:wyniki}).

\begin{table}
\centering
\caption{Opis tabeli nad nią.}
\label{id:tab:wyniki}
\begin{tabular}{rrrrrrrr}
\toprule
	         &                                     \multicolumn{7}{c}{metoda}                                      \\
	         \cmidrule{2-8}
	         &         &         &        \multicolumn{3}{c}{alg. 3}        & \multicolumn{2}{c}{alg. 4, $\gamma = 2$} \\
	         \cmidrule(r){4-6}\cmidrule(r){7-8}
	$\zeta$ &     alg. 1 &   alg. 2 & $\alpha= 1.5$ & $\alpha= 2$ & $\alpha= 3$ &   $\beta = 0.1$  &   $\beta = -0.1$ \\
\midrule
	       0 &  8.3250 & 1.45305 &       7.5791 &    14.8517 &    20.0028 & 1.16396 &                       1.1365 \\
	       5 &  0.6111 & 2.27126 &       6.9952 &    13.8560 &    18.6064 & 1.18659 &                       1.1630 \\
	      10 & 11.6126 & 2.69218 &       6.2520 &    12.5202 &    16.8278 & 1.23180 &                       1.2045 \\
	      15 &  0.5665 & 2.95046 &       5.7753 &    11.4588 &    15.4837 & 1.25131 &                       1.2614 \\
	      20 & 15.8728 & 3.07225 &       5.3071 &    10.3935 &    13.8738 & 1.25307 &                       1.2217 \\
	      25 &  0.9791 & 3.19034 &       5.4575 &     9.9533 &    13.0721 & 1.27104 &                       1.2640 \\
	      30 &  2.0228 & 3.27474 &       5.7461 &     9.7164 &    12.2637 & 1.33404 &                       1.3209 \\
	      35 & 13.4210 & 3.36086 &       6.6735 &    10.0442 &    12.0270 & 1.35385 &                       1.3059 \\
	      40 & 13.2226 & 3.36420 &       7.7248 &    10.4495 &    12.0379 & 1.34919 &                       1.2768 \\
	      45 & 12.8445 & 3.47436 &       8.5539 &    10.8552 &    12.2773 & 1.42303 &                       1.4362 \\
	      50 & 12.9245 & 3.58228 &       9.2702 &    11.2183 &    12.3990 & 1.40922 &                       1.3724 \\
\bottomrule
\end{tabular}
\end{table}  

%\chapter{[Przedmiot pracy]}
%
%\begin{itemize}
%\item rozwiązanie zaproponowane przez dyplomanta
%\item analiza teoretyczna rozwiązania
%\item uzasadnienie wyboru zastosowanych metod, algorytmów, narzędzi
%\end{itemize}
 % [Przedmiot pracy]

% TODO
%\chapter{Badania}
%
% 
%
%Rozdział przedstawia przeprowadzone badania. Jest to zasadnicza część i~musi wyraźnie dominować w~pracy.
%Badania i analizę wyników należy przeprowadzić, tak jak jest przyjęte w środowisku naukowym (na przykład korzystanie z danych benchmarkowych, walidacja krzyżowa, zapewnienie powtarzalności testów itd). 
%
%\section{Metodyka badań}
%
%\begin{itemize}
%\item opis metodyki badań
%\item opis stanowiska badawczego (opis interfejsu aplikacji badawczych -- w~załączniku)
%\end{itemize}
%
%
%\section{Zbiory danych}
%
%\begin{itemize}
%\item opis danych
%\end{itemize}
%
%
%\section{Wyniki}
%
%\begin{itemize}
%\item prezentacja wyników, opracowanie i poszerzona dyskusja  wyników, wnioski
%\end{itemize}
%
% 
%\begin{table}
%\centering
%\caption{Opis tabeli nad nią.}
%\label{id:tab:wyniki}
%\begin{tabular}{rrrrrrrr}
%\toprule
%	         &                                     \multicolumn{7}{c}{metoda}                                      \\
%	         \cmidrule{2-8}
%	         &         &         &        \multicolumn{3}{c}{alg. 3}        & \multicolumn{2}{c}{alg. 4, $\gamma = 2$} \\
%	         \cmidrule(r){4-6}\cmidrule(r){7-8}
%	$\zeta$ &     alg. 1 &   alg. 2 & $\alpha= 1.5$ & $\alpha= 2$ & $\alpha= 3$ &   $\beta = 0.1$  &   $\beta = -0.1$ \\
%\midrule
%	       0 &  8.3250 & 1.45305 &       7.5791 &    14.8517 &    20.0028 & 1.16396 &                       1.1365 \\
%	       5 &  0.6111 & 2.27126 &       6.9952 &    13.8560 &    18.6064 & 1.18659 &                       1.1630 \\
%	      10 & 11.6126 & 2.69218 &       6.2520 &    12.5202 &    16.8278 & 1.23180 &                       1.2045 \\
%	      15 &  0.5665 & 2.95046 &       5.7753 &    11.4588 &    15.4837 & 1.25131 &                       1.2614 \\
%	      20 & 15.8728 & 3.07225 &       5.3071 &    10.3935 &    13.8738 & 1.25307 &                       1.2217 \\
%	      25 &  0.9791 & 3.19034 &       5.4575 &     9.9533 &    13.0721 & 1.27104 &                       1.2640 \\
%	      30 &  2.0228 & 3.27474 &       5.7461 &     9.7164 &    12.2637 & 1.33404 &                       1.3209 \\
%	      35 & 13.4210 & 3.36086 &       6.6735 &    10.0442 &    12.0270 & 1.35385 &                       1.3059 \\
%	      40 & 13.2226 & 3.36420 &       7.7248 &    10.4495 &    12.0379 & 1.34919 &                       1.2768 \\
%	      45 & 12.8445 & 3.47436 &       8.5539 &    10.8552 &    12.2773 & 1.42303 &                       1.4362 \\
%	      50 & 12.9245 & 3.58228 &       9.2702 &    11.2183 &    12.3990 & 1.40922 &                       1.3724 \\
%\bottomrule
%\end{tabular}
%\end{table}  
%
%
% 
%\begin{figure}
%\centering
%\begin{tikzpicture}
%\begin{axis}[
%    y tick label style={
%        /pgf/number format/.cd,
%            fixed,   % po zakomentowaniu os rzednych jest indeksowana wykladniczo
%            fixed zerofill, % 1.0 zamiast 1
%            precision=1,
%        /tikz/.cd
%    },
%    x tick label style={
%        /pgf/number format/.cd,
%            fixed,
%            fixed zerofill,
%            precision=2,
%        /tikz/.cd
%    }
%]
%\addplot [domain=0.0:0.1] {rnd};
%\end{axis} 
%\end{tikzpicture}
%\caption{Podpis rysunku po rysunkiem.}
%\label{fig:2}
%\end{figure}

 % Badania

% TODO

\chapter{Podsumowanie}
\begin{itemize}
\item syntetyczny opis wykonanych prac
\item wnioski
\item możliwość rozwoju, kontynuacji prac, potencjalne nowe kierunki
\item Czy cel pracy zrealizowany? 
\end{itemize}

 % Podsumowanie

\backmatter 

%\bibliographystyle{plplain} % bibtex
%\bibliography{bibliografia} % bibtex
\printbibliography           % biblatex 
\addcontentsline{toc}{chapter}{Bibliografia}

\begin{appendices}

% TODO
\chapter{Dokumentacja techniczna}
 % dokumentacja techniczna

% TODO
\chapter{Spis skrótów i symboli}

\begin{itemize}
\item[DNA] kwas deoksyrybonukleinowy (ang. \ang{deoxyribonucleic acid})
\item[MVC] model -- widok -- kontroler (ang. \ang{model--view--controller}) 
\item[$N$] liczebność zbioru danych
\item[$\mu$] stopnień przyleżności do zbioru
\item[$\mathbb{E}$] zbiór krawędzi grafu
\item[$\mathcal{L}$] transformata Laplace'a 
\end{itemize} % Spis skrótów i symboli

% TODO
\chapter{Lista dodatkowych plików, uzupełniających tekst pracy (jeżeli dotyczy)} 

W systemie do pracy dołączono dodatkowe pliki zawierające:
\begin{itemize}
\item źródła programu,
\item zbiory danych użyte w~eksperymentach,
\item film pokazujący działanie opracowanego oprogramowania lub zaprojektowanego i wykonanego urządzenia,
\item itp.
\end{itemize}
 % Lista dodatkowych plików, uzupełniających tekst pracy – jeżeli dotyczny, w przecinym razie – zakomentuj!

\listoffigures
\addcontentsline{toc}{chapter}{Spis rysunków}
\listoftables
\addcontentsline{toc}{chapter}{Spis tabel}
	
\end{appendices}

\end{document}


%% Finis coronat opus.
